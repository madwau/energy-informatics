%
\documentclass{sig-alternate}
\usepackage[utf8]{inputenc}

\makeatletter
\def\@copyrightspace{\relax}
\makeatother
\usepackage{hyperref}

\begin{document}

\conferenceinfo{Energy Informatics}{2016 Munich, Germany}
\title{Interactive Front-End for EV Traffic Simulation in Highways}


\numberofauthors{2}

\author{
\alignauthor Adrian Thiesen\\
       \affaddr{Fakultät für Informatik}\\
       \affaddr{Technische Universität München}\\
       \email{adrian.thiesen@tum.de}
\alignauthor Martin Wauligmann\\
       \affaddr{Fakultät für Informatik}\\
       \affaddr{Technische Universität München}\\
       \email{wauligma@in.tum.de}
}


\maketitle


\begin{abstract}

As set in the OECD Environmental Outlook to 2050 \href{https://www.oecd.org/env/cc/49082173.pdf}{[1]}.
CO2 reduction will be a major target for the OECD nations in the near future. Even though electric vehicles [EV],
could play a big factor in achieving that goal, their sales are strikingly low in Germany.
Only 0.39\% of new registration in germany were EV.
\href{http://www.kba.de/DE/Statistik/Fahrzeuge/Neuzulassungen/neuzulassungen_node.html}{[2]}
A major throwback with EV is there generally low range below 150km.
To compensate that Victor del Razo and Hans-Arno Jacobsen, have outlined a method in their research paper
Smart Charging Schedules for Highway Travel with Electric Vehicles, for scheduling charging stops during
long distance travel e.g. 500km  so that the final destination is reached with the lowest cost possible,
in their case cost travel time \href{https://www.i13.in.tum.de/fileadmin/w00bof/www/PDF/TTE_2016.pdf}{[3}].
But this project is missing some interface, that shows to the driver  basic information of the process e.g.
a map of his position and charging stations. Besides that another sort of interface is desired, for someone
that manages the simulation [SMI].
Therefore this work will propose prototypes for two Front-Ends.
One for the driver, a driver interface [DI] and one for the managing persona a simulation managing interface [SMI].

\end{abstract}


\section{Introduction}

Comfortably traveling longer highway routes with an EV is currently a challenging endeavor. The charging
infrastructure is still nascent, which potentially causes already long charging times to be prolonged by
having to wait for a free spot at a charging station that runs at full capacity.

The whitepaper ``Smart Charging Schedules for Highway Travel with Electric Vehicles'' by Victor del Razo and
Hans-Arno Jacobsen first and foremost proposes a scheduling approach for EVs to decide on their charging stops during
a trip on a highway where the charging infrastructure is limited. The goal is to reduce the total travel time for
each EV. It then introduces a simulation framework that demonstrates the smart scheduling approach and on top of that
aids the simulation process by providing generated trip data and support for time-dependent parameters that effect
highway traffic.

In the following, the general idea of the scheduling approach and the resulting mathematical model will be briefly
outlined, mostly pertaining to the aspects that are relevant for the simulation tool. The problem is formulated as a
shortest path problem and the scheduling algorithm is based on the A* search algorithm that is well known in the
domain of graph traversal. The A* algorithm is extended with verification of constraints, such as EV energy
limitations and driving speeds.

The model has three main components: EVs, a highway and charging stations. The scheduling strategy is local to the EV
but requires real-time communication with charging stations and a general highway-related information system. A set
of charging stops and times is calculated and the relevant bookings are submitted to the charging stations together
with the expected arrival time. The trip proceeds as planned unless an update schedule event is received.

The EVs do not interact directly with each other, however, they indirectly influence each other via relying on the
communicated information from the charging stations on their current occupancy level and the state of their reservation
system. The reservation system supports dynamic updates as hinted at earlier.

The simulation tool implements a representation of the described model and its behavior. Additionally, it accounts for
highway exits/entries, variable highway speed limitations, and EV-specific characteristics.

Our task now is to design and implement a graphic interface that shows the states of all EVs and charging
stations on the highway as they develop during the simulation. Moreover, an interface for the EV driver, that
presents all relevant vehicle and travel related information, should be conceived.





\section{Background and Significance}
Our contribution for the project can be outlined with in the research questions stated as the follow.

\begin{enumerate}
\item 
How does a graphic interface for showing the progress of the
simulation should look like?

\item 
What tools, paradigms, etc. should be used?

\item 
What are the existing standards for integration to vehicle bus?

hello?
\end{enumerate}

\section{Preliminary Literature Review}
\section{Results outlook}
\section{Timeline}
\begin{tabular}{|c|c|c|}\hline
Objective & Schedule & Deadline\\\hline
Phase 1: Prewriting
 & 2 week & 16.11 \\\hline
First research and examination of literature sources
 & 2 weeks &  \\\hline
Research context and background & 1 week &  \\\hline
Create a 2-page preliminary statement  & 1 week &  \\\hline
Phase 2: Writing & a & a \\\hline
a & a & a \\\hline
a & a & a \\\hline
a & a & a \\\hline
Phase 3: & a & a \\\hline
a & a & a \\\hline
a & a & a \\\hline
\end{tabular}

\end{document}
