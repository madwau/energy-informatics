\documentclass[hidelinks]{sig-alternate}
\usepackage[utf8]{inputenc}
\usepackage{cite}
\usepackage{hyperref}

\makeatletter
\def\@copyrightspace{\relax}
\makeatother


\begin{document}

\conferenceinfo{Energy Informatics}{2016 Munich, Germany}
\title{Interactive Front-End for EV Traffic Simulation in Highways}


\numberofauthors{2}

\author{
\alignauthor Adrian Thiesen\\
       \affaddr{Fakultät für Informatik}\\
       \affaddr{Technische Universität München}\\
       \email{adrian.thiesen@tum.de}
\alignauthor Martin Wauligmann\\
       \affaddr{Fakultät für Informatik}\\
       \affaddr{Technische Universität München}\\
       \email{wauligma@in.tum.de}
}


\maketitle


\begin{abstract}

As set in the OECD Environmental Outlook to 2050 \href{https://www.oecd.org/env/cc/49082173.pdf}{[1]}. CO2 reduction
will be a major target for the OECD nations in the near future. Even though electric vehicles [EV], could play a big
factor in achieving that goal, their sales are strikingly low in Germany. Only 0.39\% of new registration in germany
were EV. \href{http://www.kba.de/DE/Statistik/Fahrzeuge/Neuzulassungen/neuzulassungen_node.html}{[2]} A major
throwback with EV is there generally low range below 150km. To compensate that Victor del Razo and Hans-Arno
Jacobsen, have outlined a method in their research paper Smart Charging Schedules for Highway Travel with Electric
Vehicles, for scheduling charging stops during long distance travel e.g. 500km  so that the final destination is
reached with the lowest cost possible, in their case cost travel time \href{https://www.i13.in.tum
.de/fileadmin/w00bof/www/PDF/TTE_2016.pdf}{[3}]. But this project is missing some interface, that shows to the driver
basic information of the process e.g. a map of his position and charging stations. Besides that another sort of
interface is desired, for someone that manages the simulation. Therefore this work will propose prototypes
for two Front-Ends. One for the driver, a driver user interface [DUI] and one for the managing persona a simulation
managing user interface [SMUI].

\end{abstract}


\section{Introduction}


Comfortably traveling longer highway routes with an EV is currently a challenging endeavor. The charging
infrastructure is still nascent, which potentially causes already long charging times to be prolonged by having to
wait for a free spot at a charging station that runs at full capacity.

The whitepaper ``Smart Charging Schedules for Highway Travel with Electric Vehicles'' by Victor del Razo and
Hans-Arno Jacobsen first and foremost proposes a scheduling approach for EVs to decide on their charging stops during
a trip on a highway where the charging infrastructure is limited. The goal is to reduce the total travel time for
each EV. It then introduces a simulation framework that demonstrates the smart scheduling approach and on top of that
aids the simulation process by providing generated trip data and support for time-dependent parameters that effect
highway traffic.

In the following, the general idea of the scheduling approach and the resulting mathematical model will be briefly
outlined, mostly pertaining to the aspects that are relevant for the simulation tool. The problem is formulated as a
shortest path problem and the scheduling algorithm is based on the A* search algorithm that is well known in the
domain of graph traversal. The A* algorithm is extended with verification of constraints, such as EV energy
limitations and driving speeds.

The model has three main components: EVs, a highway and charging stations. The scheduling strategy is local to the EV
but requires real-time communication with charging stations and a general highway-related information system. A set
of charging stops and times is calculated and the relevant bookings are submitted to the charging stations together
with the expected arrival time. The trip proceeds as planned unless an update schedule event is received.

The EVs do not interact directly with each other, however, they indirectly influence each other via relying on the
communicated information from the charging stations on their current occupancy level and the state of their reservation
system. The reservation system supports dynamic updates as hinted at earlier.

The simulation tool implements a representation of the described model and its behavior. Additionally, it accounts for
highway exits/entries, variable highway speed limitations, and EV-specific characteristics.

Our task now is to design and implement a graphic interface that shows the states of all EVs and charging stations on
the highway as they develop during the simulation. Moreover, an interface for the EV driver, that presents all
relevant vehicle and travel related information, should be conceived.


\section{Research}

Our main research question is as follows: What is the most suitable form of presentation for the data that is most
relevant during the simulation and while driving respectively?

There are a variety of aspects that need to be considered when designing a frontend for data-heavy applications,
such as the highway traffic simulation we are dealing with.


\section{Requirements}

Driving user interfaces have often proven to be complicated or not so well integrated, therefore in our approach we
will try to keep it as minimalistic as possible but still have all necessary features onboard. Most important for the
driver UI are the self-stats like time, current position of the car, probably integrated as a map with a display of
possible charging stations around. The time and distance already traveled and the current speed of the vehicle.
Furthermore, EV specific statuses, like the time waited for charging, the current battery level as well as an
estimate of charging time and specific to the simulation the schedule status. Which means that the user will always
be able to tell what the battery status is, his distance to the destination as well as the distance to the next
scheduled stop, so that he has an overview of the remaining trip. Besides the display of statuses, a possibility
will be integrated, so the user can enter a destination, so that the system can then calculate the optimal charging
route. In addition, it will then display the route and show the user scheduled stops. While the user is driving,
there will be some sort of notification, so that he is informed in time, when he is getting close to a stop.

The simulation user interface will show an overview of all charging stations. It will give detailed information for
each charging station, so that he can follow the stations development. Information about the scheduling algorithm
will be the main focus here, like queue length queue prediction of waiting cars, busy poles, arrived and leaved cars
and cars that a currently plugged in. Furthermore, information about energy management will also play a key role,
like energy consumed, energy produced, energy stored and energy bought. Another information that will be provided is
closer information about a current charge event like the EV the duration used charging technology etc. A main focuse
for the simulation user interface will be, providing aggregated and cumulative statistics about EV and charging
stations.

\section{Preliminary Literature Review}
Automotive User Interfaces have been around since decades, consequently our goal will not be to reinvent the wheel but to examine existing standards and stick to those. Especially the once ISO – International Organization for standardization offers are a good source, when it comes to recommendations and best practice. The once defining an automotive user interface are worth mentioning. 

\noindent
\def\arraystretch{1.2}
\begin{tabular}{p{0.45\linewidth} | p{0.45\linewidth}}
Standard & Description \\\hline
ISO 15005:2002 & Dialogue management principles and compliance procedures [\href{http://www.iso.org/iso/home/store/catalogue_tc/catalogue_detail.htm?csnumber=34085}{4}]\\\hline
ISO 15006:2011 & Specifications for in-vehicle auditory presentation [\href{http://www.iso.org/iso/home/store/catalogue_tc/catalogue_detail.htm?csnumber=55322}{5}]\\\hline
ISO 15007-1:2014 2:2014 & Measurement of driver visual behaviour [\href{http://www.iso.org/iso/home/store/catalogue_tc/catalogue_detail.htm?csnumber=63220}{6}]\\\hline
ISO/TR 16352:2005 & Warning systems [\href{http://www.iso.org/iso/home/store/catalogue_tc/catalogue_detail.htm?csnumber=37859}{7}]\\\hline
ISO 16673:2007 & Occlusion method to assess visual distraction [\href{http://www.iso.org/iso/home/store/catalogue_tc/catalogue_detail.htm?csnumber=38035\#5}{8}]\\\hline
ISO/TS 16951:2004 & on-board message priority [\href{http://www.iso.org/iso/home/store/catalogue_tc/catalogue_detail.htm?csnumber=29024}{9}] \\\hline
ISO 26022:2010 & Simulated lane change test to assess distraction [\href{http://www.iso.org/iso/home/store/catalogue_tc/catalogue_detail.htm?csnumber=43361}{10}]
\end{tabular}
Obtaining those standards is not easy and cost intensive therefore it is unsure at this point how and if we can get access to them. Design knowledge, when it comes to  user interfaces, plays a key role, especially in vehicles, where distraction is a big factor  \cite{Developmentofanautomotiveuserinterfacedesignknowledgesystem}.
 Another thing that needs to be taken into consideration in vehicles is situation awareness \cite{Skrypchuk2016}. A high challenge currently is compositing different heterogeneous User interfaces into existing systems \cite{Holstein2015}. Which also apply's to our prototype, since we also have to have in mind that it might be integrated into an existing System in the future.


\bibliography{references}
\bibliographystyle{IEEEtran}
\end{document}
