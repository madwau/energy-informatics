\begin{frame}
\frametitle{Aufzählung}

Bei kleinen Aufzählungen auf Aufzählungszeichen verzichten und ggf. zusätzliche Leerzeile.\newline
Nur die wesentlichen Punkte nennen und Themen auf verschiedene Seiten splitten.\\
Punkt 1\\
Punkt 2

Wenn Unterpunkte in einer Aufzählung nötig sind ist ein Einrücken mit \PraesentationAufzaehlungEbeneZweiSymbol{} möglich

\begin{PraesentationAufzaehlung}
    \item Unterpunkt 1
        \begin{itemize}
            \item Unterpunkt 1
            \item Unterpunkt 2
        \end{itemize}
\end{PraesentationAufzaehlung}

Bei größeren Listen die Standardeinstellung \PraesentationAufzaehlungEbeneEinsSymbol{} verwenden

\begin{PraesentationAufzaehlung}
    \item Unterpunkt 1
    \item Unterpunkt 2
    \item Unterpunkt 3
\end{PraesentationAufzaehlung}

\end{frame}
\clearpage



\begin{frame}
\frametitle{Aufzählung}

Bei kleinen Aufzählungen auf Aufzählungszeichen verzichten und ggf. zusätzliche Leerzeile.\newline
Nur die wesentlichen Punkte nennen und Themen auf verschiedene Seiten splitten.\\
Punkt 1\\
Punkt 2

Wenn Unterpunkte in einer Aufzählung nötig sind ist ein Einrücken mit \PraesentationAufzaehlungEbeneZweiSymbol{} möglich

\begin{PraesentationAufzaehlung}
    \item Unterpunkt 1
        \begin{itemize}
            \item Unterpunkt 1
            \item Unterpunkt 2
        \end{itemize}
\end{PraesentationAufzaehlung}

Bei größeren Listen die Standardeinstellung \PraesentationAufzaehlungEbeneEinsSymbol{} verwenden

\begin{PraesentationAufzaehlung}
    \item Unterpunkt 1
    \item Unterpunkt 2
    \item Unterpunkt 3
\end{PraesentationAufzaehlung}

\end{frame}
\clearpage